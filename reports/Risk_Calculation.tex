\documentclass[11pt]{article}

\usepackage[utf8]{inputenc} % usually default in Overleaf, but safe
\usepackage[T1]{fontenc}
\usepackage{lmodern}
\usepackage{amsmath, amssymb}
\usepackage{geometry}
\usepackage{url}
\geometry{margin=1in}

\title{Dynamic Accident Risk Model for Citi Bike Trips in NYC}
\author{}
\date{}

\begin{document}
\maketitle

\section{Goal}
We build a \textbf{dynamic risk model} that outputs the probability (in \%) that a Citi Bike trip results
in an accident, conditional on the rider starting at a specific station and the current time. This
estimate can be used to derive a \textbf{dynamic per-trip health insurance price}.

\section{Notation}
\begin{itemize}
    \item $S$: set of all stations.
    \item $s \in S$: a station.
    \item $s_{\mathrm{cur}} \in S$: current (start) station.
    \item $s_{\mathrm{dest}} \in S$: destination station.
    \item $r$: a (random) trip.
    \item $r_{s_1 \to s_2}$: trip starting at $s_1$ and ending at $s_2$.
    \item $t_r$: time window associated with trip $r$ (e.g.\ start time or start-to-end interval).
    \item We write $P(\cdot)$ for probabilities of events. For continuous variables, we write $f(\cdot)$ for probability density functions (pdfs).
\end{itemize}

\section{Model decomposition}
For a fixed start station $s_{\mathrm{cur}}$:
\begin{align}
P(\mathrm{accident} \mid s_{\mathrm{cur}})
&= \sum_{s_{\mathrm{dest}} \in S \setminus \{s_{\mathrm{cur}}\}}
P\!\big(\mathrm{accident}, r_{s_{\mathrm{cur}}\to s_{\mathrm{dest}}} \,\big|\, s_{\mathrm{cur}}\big) \\
&= \sum_{s_{\mathrm{dest}} \in S \setminus \{s_{\mathrm{cur}}\}}
P\!\big(r_{s_{\mathrm{cur}}\to s_{\mathrm{dest}}} \mid s_{\mathrm{cur}}\big)\,
P\!\big(\mathrm{accident} \mid r_{s_{\mathrm{cur}}\to s_{\mathrm{dest}}}\big).
\end{align}
The second line follows from the \textbf{law of total probability} and the product rule.

This yields two subproblems:
\begin{enumerate}
    \item Destination choice:
    \[
    P\!\big(r_{s_{\mathrm{cur}}\to s_{\mathrm{dest}}} \mid s_{\mathrm{cur}}\big).
    \]
    \item Conditional trip risk:
    \[
    P\!\big(\mathrm{accident} \mid r_{s_{\mathrm{cur}}\to s_{\mathrm{dest}}}\big).
    \]
\end{enumerate}

\section{Destination choice}
\paragraph{Reference.} Notebook \verb|03_citibike_FE_Modelling| (trained for a fixed $s_{\mathrm{cur}}$).
\[
P\!\big(r_{s_{\mathrm{cur}}\to s_{\mathrm{dest}}} \mid s_{\mathrm{cur}}\big).
\]

We model the destination distribution with a probabilistic classifier (multinomial logistic regression), producing predicted probabilities for each candidate destination.

\paragraph{Issue: class explosion.}
The number of possible destination stations is large (on the order of $\sim 2000$), which makes a direct multi-class formulation noisy and data-inefficient.

\paragraph{Mitigation.}
We restrict the target space to the $k$ most frequent destination stations (calculated for each $s_{\mathrm{cur}}$ individually) and collapse all remaining stations into a single ``Other'' class:
\[
\mathcal{C} = \{s^{(1)}, \dots, s^{(k)}, \mathrm{Other}\}.
\]

\section{Conditional trip risk}
We approximate the conditional accident probability as a ratio of expected accident count to expected rider count on the route during the relevant time window:
\begin{equation}
P(\mathrm{accident} \mid r_{s_{\mathrm{cur}}\to s_{\mathrm{dest}}})
\approx
\frac{\mathbb{E}\!\left[N_{\mathrm{acc}}\!\left(r_{s_{\mathrm{cur}}\to s_{\mathrm{dest}}}\right)\right]}
{\mathbb{E}\!\left[N_{\mathrm{riders}}\!\left(r_{s_{\mathrm{cur}}\to s_{\mathrm{dest}}}\right)\right]}.
\end{equation}

\subsection{Expected accident count on a route}
We decompose the expected accident count into (notation: $r=r_{s_{\mathrm{cur}}\to s_{\mathrm{dest}}}$):
\begin{equation}
\mathbb{E}\!\left[N_{\mathrm{acc}}\!\left(r_{s_{\mathrm{cur}}\to s_{\mathrm{dest}}}\right)\right]
=
N_{\mathrm{acc,day}}\!\big(d(t_r)\big)
\cdot
\int_{\mathcal{R}(x_r,t_r)} f_{\mathrm{acc}}(x,\tau)\,dx\,d\tau,
\end{equation}
where $d(t_r)$ is the calendar day of the trip window, $x=(\mathrm{lat},\mathrm{lng})$ is location, $\tau$ is time-of-day, and $\mathcal{R}(x_r,t_r)$ denotes the spatio-temporal ``tube'' of the trip.

\subsubsection{Daily accident volume}
\paragraph{Reference.} \verb|MV_Collision| notebook.
\[
N_{\mathrm{acc,day}}(d).
\]
We predict the total number of accidents per day using regression models (Linear Regression, XGBoost, Random Forest) and a boosted hybrid approach (linear model + XGBoost on residuals).

\subsubsection{Spatio-temporal accident intensity}
\[
f_{\mathrm{acc}}(x,\tau).
\]
We model accident intensity over location and time-of-day via a \textbf{mixture of Gaussians (MoG)} fitted to all accidents with injured cyclists aggregated by $(\mathrm{lat},\mathrm{lng},\tau)$.

\paragraph{Model assumption.} The accident distribution in New York only depends on time of day.

\subsection{Expected rider count on a route}
Analogously:
\begin{equation}
\mathbb{E}\!\left[N_{\mathrm{riders}}\!\left(r_{s_{\mathrm{cur}}\to s_{\mathrm{dest}}}\right)\right]
=
\bar{N}_{\mathrm{riders,NYC}}(t_r)
\cdot
\int_{\mathcal{R}(x_r)} f_{\mathrm{riders}}(x)\,dx.
\end{equation}

\paragraph{Model assumption.} We assume that the traffic distribution is stationary for easier modelling.

\subsubsection{Citywide rider volume}
We approximate citywide rider volume by the number of active Citi Bike rides at time of the trip start, scaled by a constant factor (there are bicycle riders which aren't riding Citi Bikes). Since the average ride duration is short ($T \approx 12$ minutes), the scaled active rider count is a reasonable proxy:
\[
\bar{N}_{\mathrm{riders}}(t_r)
\approx
N_{\mathrm{citibike\_riders}}(t_{\mathrm{start}})\cdot c_{\mathrm{scaling}}.
\]
We estimate the scaling constant $c$ as
\[
c
= \frac{\text{Total bicycle rides in New York City in 2025}}
{\text{Total Citi Bike trips in New York City in 2025}} \, ,
\]
using Citi Bike data and NYC bicycle statistics (\url{https://www.nyc.gov/html/dot/html/bicyclists/bikestats.shtml}).

\subsubsection{Spatio-temporal rider intensity}
For each trip we map start and end coordinates on the map and fit a \textbf{mixture of Gaussians (MoG)} to obtain a normalized rider density:
\[
f_{\mathrm{riders}}(x).
\]

\subsection{Approximating Conditional trip risk}
\begin{equation}
P(\mathrm{accident} \mid r_{s_{\mathrm{cur}}\to s_{\mathrm{dest}}})
\approx
\frac{\mathbb{E}\!\left[N_{\mathrm{acc}}\!\left(r_{s_{\mathrm{cur}}\to s_{\mathrm{dest}}}\right)\right]}
{\mathbb{E}\!\left[N_{\mathrm{riders}}\!\left(r_{s_{\mathrm{cur}}\to s_{\mathrm{dest}}}\right)\right]}.
\end{equation}
\begin{equation}
\approx
\frac{
N_{\mathrm{acc,day}}\!\big(d(t_r)\big)\cdot
\int_{\mathcal{R}(x_r,t_r)} f_{\mathrm{acc}}(x,\tau)\,dx\,d\tau
}{
N_{\mathrm{riders,NYC}}(t_r)\cdot
\int_{\mathcal{R}(x_r)} f_{\mathrm{riders}}(x)\,dx
}
\end{equation}
\begin{equation}
=
\frac{N_{\mathrm{acc,day}}\!\big(d(t_r)\big)}{N_{\mathrm{riders,NYC}}(t_r)}
\cdot
\frac{\int_{\mathcal{R}(x_r,t_r)} f_{\mathrm{acc}}(x,\tau)\,dx\,d\tau}
{\int_{\mathcal{R}(x_r)} f_{\mathrm{riders}}(x)\,dx}
\end{equation}

Assume that the accident density function will not be time dependent for this duration of the trip (stop integrating over it, multiply with the constant (trip duration) and a representative value).
\begin{equation}
\approx
\frac{N_{\mathrm{acc,day}}\!\big(d(t_r)\big)}{N_{\mathrm{riders,NYC}}(t_r)}
\cdot
\frac{\Delta\tau \int_{\mathcal{R}(x_r)} f_{\mathrm{acc}}(x,\bar{\tau}_r)\,dx}
{\int_{\mathcal{R}(x_r)} f_{\mathrm{riders}}(x)\,dx}
\end{equation}

Now split $\mathcal{R}(x_r)$ into $n$ regions $\mathcal{R}_1,\dots,\mathcal{R}_n$ of equal size.
\begin{equation}
\approx
\frac{N_{\mathrm{acc,day}}\!\big(d(t_r)\big)}{N_{\mathrm{riders,NYC}}(t_r)}
\cdot
\Delta\tau
\cdot
\frac{ \sum_{i=1}^{n}\int_{\mathcal{R}_i} f_{\mathrm{acc}}(x,\bar{\tau}_r)\,dx}
{\sum_{i=1}^{n}\int_{\mathcal{R}_i} f_{\mathrm{riders}}(x)\,dx}
\end{equation}

Assume that $f_{\mathrm{riders}}$ and $f_{\mathrm{acc}}$ are constant on the regions.
\begin{equation}
\approx
\frac{N_{\mathrm{acc,day}}\!\big(d(t_r)\big)}{N_{\mathrm{riders,NYC}}(t_r)}
\cdot
\Delta\tau
\cdot
\frac{ \sum_{i=1}^{n} \Delta\mathcal{R}_i\, f_{\mathrm{acc}}(\bar{x}_i,\bar{\tau}_r)}
{\sum_{i=1}^{n} \Delta\mathcal{R}_i\, f_{\mathrm{riders}}(\bar{x}_i)}
\end{equation}

By definition, all $\Delta \mathcal{R}_i$ are of equal size.
\begin{equation}
\approx
\frac{N_{\mathrm{acc,day}}\!\big(d(t_r)\big)}{N_{\mathrm{riders,NYC}}(t_r)}
\cdot
\Delta\tau
\cdot
\frac{ \sum_{i=1}^{n} f_{\mathrm{acc}}(\bar{x}_i,\bar{\tau}_r)}
{\sum_{i=1}^{n} f_{\mathrm{riders}}(\bar{x}_i)}
\end{equation}

\section{Final Model}
The final dynamic risk estimate is:
\begin{align}
P(\mathrm{accident} \mid s_{\mathrm{cur}})
&= \sum_{s_{\mathrm{dest}} \in S \setminus \{s_{\mathrm{cur}}\}}
P\!\big(r_{s_{\mathrm{cur}}\to s_{\mathrm{dest}}} \mid s_{\mathrm{cur}}\big)\,
P\!\big(\mathrm{accident} \mid r_{s_{\mathrm{cur}}\to s_{\mathrm{dest}}}\big) \\
&= \sum_{s_{\mathrm{dest}} \in S \setminus \{s_{\mathrm{cur}}\}}
P\!\big(r_{s_{\mathrm{cur}}\to s_{\mathrm{dest}}} \mid s_{\mathrm{cur}}\big)
\cdot
\frac{N_{\mathrm{acc,day}}\!\big(d(t_r)\big)}{N_{\mathrm{riders,NYC}}(t_r)}
\cdot
\Delta\tau
\cdot
\frac{ \sum_{i=1}^{n} f_{\mathrm{acc}}(\bar{x}_i,\bar{\tau}_r)}
{\sum_{i=1}^{n} f_{\mathrm{riders}}(\bar{x}_i)} \\
&= \Delta\tau
\cdot
\frac{N_{\mathrm{acc,day}}\!\big(d(t_r)\big)}{N_{\mathrm{citibike\_riders}}(t_{\mathrm{start}})\cdot c_{\mathrm{scaling}}}
\cdot
\sum_{s_{\mathrm{dest}} \in S \setminus \{s_{\mathrm{cur}}\}}
P\!\big(r_{s_{\mathrm{cur}}\to s_{\mathrm{dest}}} \mid s_{\mathrm{cur}}\big)
\cdot
\frac{ \sum_{i=1}^{n} f_{\mathrm{acc}}(\bar{x}_i,\bar{\tau}_r)}
{\sum_{i=1}^{n} f_{\mathrm{riders}}(\bar{x}_i)}.
\end{align}
Now we have everything needed to compute this:
\begin{itemize}
  \item $\Delta\tau$: average trip duration (02\_Citi\_Bike\_EDA)
  \item $N_{\mathrm{acc,day}}\!\big(d(t_r)\big)$: predicted daily number of accidents from a linear regression model (MV\_Collision)
  \item $N_{\mathrm{citibike\_riders}}(t_{\mathrm{start}})$: current number of active rides (citi bike live data feed)
  \item $P\!\big(r_{s_{\mathrm{cur}}\to s_{\mathrm{dest}}} \mid s_{\mathrm{cur}}\big)$:
probability that a new trip ends at destination $s_{\mathrm{dest}}$, modelled by logistic regression (\texttt{03\_Citibike\_FE\_Modelling})
\item $\bar{x}_i$: take them equally spaced from a straight line between $s_{\mathrm{cur}}$ and $s_{\mathrm{dest}}$
\item $\bar{\tau}_r$: calculate trip start + $1/2$ average trip duration 
\item $f_{\mathrm{acc}}$: MoG on bike accident data with features lat,long, time of day (to do)
\item $f_{\mathrm{riders}}$: MoG on citibike trips, from each data point use start and end coordinates (to do)
\end{itemize}

\section{Pricing Health Insurance}
We can directly calculate the insurer's expected cost for a ride starting at $s_{\mathrm{cur}}$:
\begin{equation}
\mathbb{E}\!\left[\mathrm{Cost} \mid s_{\mathrm{cur}}\right]
=
\mathbb{P}\!\left(\mathrm{acc} \mid s_{\mathrm{cur}}\right)\,
\mathbb{E}\!\left[\mathrm{Insurance\ Payout} \mid \mathrm{acc}\right].
\end{equation}
Adding a premium and we already have the dynamic price for the customer:
\begin{equation}
\mathrm{Price} = (1 + \lambda)\,\mathbb{E}\!\left[\mathrm{Cost} \mid s_{\mathrm{cur}}\right].
\end{equation}

\end{document}
